% !TeX root=../main.tex
\chapter{نتیجه گیری}

با شناسایی و اعتبارسنجی مدل سیستم و پیاده سازی کنترلر های کلاسیک نظیر PID و کنترلر های مبتنی بر پیش بینی مدل نظیر $Linear MPC$ و $Tube MPC$ در بخش های پیشین، توانستیم درک درستی از رفتار سیستم به دست آورده و نحوه ی پاسخ ددهی آن را در شرایط مختلف بررسی کنیم.
همانطور که مشاهده شد، کنترلر PID با ضرایب بالا، می تواند سیستم را به خوبی کنترل کند، اگرچه این امر مستلزم توانایی سیستم در پشتیبانی از فرمان های کنترلی زیاد است. در غیر این صورت، پاسخ مطلوب سیستم حاصل نمی شود. 
علاوه بر این، با استفاده از کنترلر MPC خطی، مشاهده کردیم که جز در مواردی مانند کنترل جهت متحرم در صفحه، کنترل سایر حالت های سیستم به سختی انجام می گیرد. مشاهده شد که کنترل ارتفاع سیستم از خاصیتی غیرخطی برخوردار است که تنها به وسیله ی کنترل های بدون اتکا بر مدل سیستم قابل ردگیری است و کنترلر MPC که مدل سیستم را در اختیار دارد، در عمل قادر به کنترل این حالت نیست. 
با افزودن کنترلر PID به کنترلر MPC، مشاهده شد که سیتسم می تواند با تلاش کنترلی کمتری نسبت به PID حالت ها را کنترل کند. اگرجه، برای حالت هایی که MPC قادر به کنترل آنها نبود ضرایب PID تغییری نکرده اند و در سایر موارد به طور قابل توجهی کاهش یافته اند. 
در نهایت، با اعمال نیرو و گشتاور اغتشاش به سیستم مشاهده شد که کنترلر های شامل المان PID توانستند به خوبی رفتار سیستم را کنترل کنند، اگرچه حساسیت سیستم به گشتاور های اغتشاشی بسیار بیشتر از حساسیت آن به نیروهای اغتشاشی است و پاسخ سیستم به مقدار قابل توجهی از مقدار رفرنس خارج می شود. با این حال، این اغتشاش در شبیه سازی منجر به ناپایداری سیستم نشده است.

\section{کارهای آینده}
با اتمام این پژوهش، مدل صحیح و معتبری از سیستم به دست آمده است که می تواند مورد پژوهش های بیشتری در زمینه های بهینه سازی کنترلر های کلاسیک، پیاده سازی کنترلر های مدرن و یا مبتنی بر هوش مصنوعی قرار گیرد. علاوه بر این، به منظور شباهت هر چه بیشتر این مدل با سیستم واقعی، می توان محدودیت های فیزیکی که در واقعیت محدودکننده عملکرد سیستم هستند را در طراحی مدل لحاظ کرد. 
در نهایت، با در اختیار داشتن سیستم فیزیکی و واقعی از سیستم مورد بررسی در این پژوهش، می توان پس از تعیین پارامتر های مورد استفاده مطابق با سیستم واقعی و اعتبارسنچی رفتار سیستم، به صورت عملیاتی سیستم را کنترل کرد.












