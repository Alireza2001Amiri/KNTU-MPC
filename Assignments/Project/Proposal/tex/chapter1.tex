% !TeX root=../main.tex

\chapter{مقدمه}
% دستور زیر باعث عدم‌نمایش شماره صفحه در اولین صفحهٔ این فصل می‌شود.
%\thispagestyle{empty}
در این فصل، ابتدا مقدمه‌ای درباره‌ی ساختارهای موتورهای صفحه ای مبتنی بر شناوری مغناطیسی معرفی شده و با توجه به ویژگی‌های منحصربه‌فرد این فناوری، کاربردهای آن در سایر صنایع نیز مورد بحث قرار می‌گیرد.


\section{مقدمه‌ای بر موتورهای صفحه‌ای مبتنی‌ بر شناوری مغناطیسی}

شناوری مغناطیسی به معنای اعمال نیروهای مغناطیسی به اجسام به‌گونه‌ای است که این نیروها بتوانند بر نیروی جاذبه غلبه کرده و جسم را بدون تماس فیزیکی و به‌صورت پایدار در هوا معلق نگه ‌دارند. این نیرو می‌تواند به دو شکل جاذبه یا دافعه اعمال شود. در حالت جاذبه‌، نیروی مغناطیسی از بالا به جسم وارد شده و نیروی گرانش زمین را خنثی می‌کند، درحالی‌ که در حالت دافعه، نیرو از پایین به جسم وارد شده و آن را به سمت بالا دفع می‌کند. در صورتی‌که جسم فقط دارای خاصیت رسانایی باشد، تنها امکان جذب‌شدن وجود دارد، اما اگر جسم از مواد مغناطیسی مانند آهنرباهای دائمی یا الکتریکی ساخته شود، می‌تواند هم جذب و هم دفع شود.

کنترل نیروهای مغناطیسی معمولاً با استفاده از آهنرباهای الکتریکی انجام می‌شود، به‌طوری ‌که عبور جریان الکتریکی از سیم‌پیچ‌ها میدان مغناطیسی ایجاد کرده و تنظیم این جریان‌ها باعث تغییر در شدت میدان و نیروی وارده به جسم می‌شود. از این طریق، می‌توان با کنترل دقیق جریان، جسم را به‌طور پایدار در حالت معلق نگه داشت.

در کاربردهای صنعتی، به‌دلیل نیاز به بازدهی بالاتر در تبدیل انرژی مغناطیسی به نیرو، از آرایه‌های خاصی از آهنرباهای دائمی به نام 
\textit{آرایه هالباخ}
\LTRfootnote{Halbach array}
استفاده می‌شود. این آرایه‌ها به‌گونه‌ای طراحی شده‌اند که میدان مغناطیسی را به‌طور متمرکز در یک سمت تقویت کنند و در نتیجه، نیروی مغناطیسی بیشتری به جسم وارد شود. ساختارهای آرایه هالباخ یک‌بعدی و دوبعدی در تحقیقات پیشین به‌طور گسترده بررسی و استفاده شده‌اند.

برای پیاده‌سازی موفق یک سیستم شناوری مغناطیسی، عوامل متعددی باید در نظر گرفته شوند که شامل طراحی و بهینه‌سازی ساختار مکانیکی سیستم، پیاده‌سازی کنترلرهای دقیق برای تنظیم نیروهای مغناطیسی، و همچنین مدل‌سازی دینامیکی یا شناسایی رفتار سیستم برای کنترل بهتر آن است. این عوامل به‌طور مستقیم بر کارایی و پایداری سیستم تأثیر می‌گذارند و باید به‌دقت مورد بررسی و تنظیم قرار گیرند.

